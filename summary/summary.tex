\chapter*{Summary}
\addcontentsline{toc}{chapter}{Summary}
\setheader{Summary}

At first glance nanopores may appear simple, almost intuitive, to understand given that they are, quite literally, `just' very small pores in a membrane. In fact, one may even wonder why at all we need trained scientists to study such seemingly simple entities. The short answer is that nanopores, as the word suggests, are nanoscale entities and, as such, one can not directly see or experience any of the events that occur down there. The long answer can be found in this thesis. Here, I present and discuss a wide array of nanopores, from biological nanopores like the nuclear pore complex (NPC), to solid-state nanopores, and DNA-origami nanopores. While the central focus of my research is to understand the inner workings of the NPC, a short journey into the world of ion transport in solid-state nanopores is first undertaken, with special emphasis on the random fluctuations of the ion flow within the nanopore, referred to as current noise. Next, I introduce the concept of biomimetic nanopores, where a solid-state nanopore is `camouflaged' by coating its inner surface with purified proteins, resulting in an entity that behaves somewhat like a real NPC. Biomimetic nanopores have enabled us to mimic, study, and gain new insights into how the real NPC works, and bear great potential for further developments and discoveries.\\[0.5pt]

\noindent \textbf{Chapter 1} provides a general introduction to the basic concepts used this thesis. Starting from a short overview of the cellular organization, I narrow down to the NPC, from the first discoveries of its architecture and biochemical composition, to the key role of FG-Nups in establishing the selective barrier. The current theories of nuclear transport are also illustrated, with emphasis on the two opposing 'FG-centric' and 'Kap-centric' models. Our experiments take advantage of many tools from nanotechnology. Therefore, I first introduce the nanopore technology, outlining common techniques of fabrication as well as illustrating the basic principles of single-molecule sensing. Following up, I present how engineered solid-state nanopores with FG-Nups have been shown to recapitulate NPC-like selective transport \emph{in vitro}. I end up by explaining the basic principles behind the DNA origami technique, and highlight important literature contributions to the field of DNA-origami nanopores.\\[0.5pt]

\noindent \textbf{Chapter 2} features a side-by-side comparison of the noise characteristics and performances of biological \emph{vs} solid-state nanopores. We start by introducing the various types of low- and high-frequency noise sources in the two systems. While they both suffer from the same noise sources at high frequencies, namely dielectric and capacitive noise, at low frequencies solid-state nanopores are largely dominated by 1/f noise, whereas biological pores typically feature protonation noise. We move on to confront the performances of a few selected nanopore systems in terms of signal-to-noise ratio (SNR) for free translocations of short homopolymers. We collected data both from literature, our own lab, as well as external labs, and found that SiN\textsubscript{x} solid-state pores featured the highest SNR ($\sim$37), which results from the combination of higher currents, hence higher signal, and low high-frequency noise that are attainable in such system. We note however that introducing a slow-down mechanism using a DNA-motor protein to feed the DNA molecule into the pore, as it was shown for the biological pore MspA, was shown to boost the SNR by >160-fold. We conclude with an overview of the most notable approaches to lower the noise at low and high frequency, for both biological and solid-state nanopores.\\[0.5pt]

\noindent To this date, there is still one noise source that puzzles many nanopore researchers, namely the 1/f noise. In \textbf{Chapter 3} we characterize the low-frequency spectrum of our solid-state nanopores and find that the magnitude the of 1/f noise fluctuations decreases for larger pores. While such behavior is interesting and appeared as well in previous works, no theoretical model has yet been able to express such trend analytically. We find that including the access region contribution into the picture, which had been dismissed in previous models, is sufficient to fit the general observed trend. We further adopt two different Hooge parameters for surface and bulk fluctuations, which reflect the presence of two different mechanisms of ion fluctuations occurring near the pore surface and in the pore bulk and access regions, respectively. Our model represents a generalization of the Hooge's model for solid-state nanopores that works well for varying nanopore diameters. \\[0.5pt]

\noindent \textbf{Chapter 4} provides a general overview of the nanopore field, with particular emphasis on nanopore applications beyond sequencing of DNA or RNA. By listing major contributions to the nanopore field from different areas of expertise, we showcase the high versatility of both biological and solid-state nanopore platforms and their applicability to solve a broad range of challenges. The high versatility of nanopores stems from the possibility to control the pore geometry, such as diameter and length, as well as to engineer the pore surface to tune and customize the interaction with the analyte molecules. 
In this review chapter, we cover a range of applications that involve both biological and solid-state nanopores, from single-molecule proteomics, single-molecule liquid biopsy for detection of relevant biomarkers, creation of nanoreactors to study polymers in confinement, to biomimetic nanopores to mimic and understand complex biological processes. In this thesis, we employ such biomimetic nanopores to investigate the physical principles of the NPC selective barrier. \\[0.5pt]

\noindent Building on previous works that showed the successful reconstitution of biomimetic nuclear pore complexes using purified native FG-Nups and solid-state nanopores, we take a step forward in \textbf{Chapter 5} and reconstitute fully artificial biomimetic NPCs in solid-state nanopores using a rationally designed FG-Nup protein that we coined `NupX'. We hypothesise that building a designer protein following a few simple design rules is sufficient for recapitulating a functional, selective biomimetic NPC. From studying the sequence of native yeast GLFG-Nups, known to be essential for proper nuclear transport and cell viability, we identify three recurring properties: (i) a bimodal distribution of amino acids featuring a N-terminal cohesive, low charge domain, and a C-terminal repulsive, high charge domain; (ii) regularly spaced FG and GLFG repeats in the cohesive domain, at intervals of $\sim5-20$ amino acid spacers; (iii) intrinsically disordered sequences. 

\noindent To assess the selective properties of our designed sequence, we reconstitute NupX brushes on QCM-D and monitor binding to increasing concentrations of transporter Kap95, finding that Kap95 binds efficiently to NupX in a concentration-dependent manner, similar to native FG-Nups. Conversely, flushing inert BSA yielded no detectable binding to NupX, indicating that NupX binds selectively to Kap95. Next, we reconstituted NupX in biomimetic nanopores and tested the selective transport properties of the reconstituted NupX-mesh. We find that, while flushing transporter Kap95 results in frequent current spikes that indicates efficient transportation of the protein through the pore, inert BSA is virtually blocked as it yields only few sporadic translocation events. Experimental work is complemented by molecular dynamics simulations that provide microscopic insights into the NupX-brushes conformation and NupX-mesh density distribution within the pore. Altogether, our data show the successful recapitulation of nuclear transport selectivity using a rationally designed protein.\\[0.5pt]

\noindent In \textbf{Chapter 6} we study the interaction between transporter Kap95 and a native FG-nucleo\-porin, Nsp1, using biomimetic nanopores. While mere selective properties of the reconstituted Nsp1-mesh have been characterized in earlier studies, we here assess the behavior of the Nsp1-mesh as a function of Kap95 concentration with the aim to provide new evidence that aids in discriminating between the current models of transport. First, we create biomimetic nanopores by grafting Nsp1 proteins to the inner wall of a solid-state nanopore. Upon flushing Kap95 in the bulk solution surrounding the pore, we observe a steady-state decrease of the ionic current, indicating that a portion of Kap95 molecules binds with high affinity to the Nsp1-mesh. Furthermore, fast ($\sim$millisecond) conductance blockades are observed on top of the current decrease, indicating the presence of a second population of fast translocating Kap95 molecules. We analyse the low-frequency noise of the ionic current, which in biomimetic nanopores is dominated by the FG-Nup fluctuations within the pore, and find a progressive decrease of the 1/f noise at increasing Kap95 concentration, which is consistent with an increase in overall rigidity of the FG-mesh. We corroborate our electrical measurements with QCM-D, where, similar to previous reports, we observe a concentration-dependent incorporation of Kap95 into the Nsp1 brush, consistent with the nanopore experiments. To sum up, our data support the presence of two distinct Kap populations within the FG-mesh, one slow and one fast, which is consistent with predictions from Kap-centric models of transport.\\[0.5pt]

\noindent The search for improved NPC mimics led scientists to develop DNA-origami scaffolds to accommodate and spatially organize FG-Nups in a controlled fashion. While NPC mimics based on DNA origami have been reported, it has not been possible, so far, to assess the transport properties of the reconstituted FG-mesh. In \textbf{Chapter 7} we address this important issue and work out a novel method to embed very large 30nm-wide DNA-origami pores across the membrane of giant liposomes. To overcome the high energy barrier for insertion of large objects into the lipid bilayer, we adopt an inverted-emulsion cDICE technique which allows us to incorporate the pores into the vesicle membrane during the process of membrane formation. We assess the correct insertion of the pores by studying the influx of different fluorescent molecules into the vesicle, finding that a range of molecules can pass through the pore, from the smaller $\sim$30 kDa protein GFP, to large 250 kDa ($\sim$28 nm) dextran molecules, whereas larger dextran molecules are excluded, consistently with the pore size cut-off. Furthermore, we quantify the number of pores per vesicle using a FRAP assay and modelling the diffusion of GFP through the pores, finding that up to hundreds of pores can be functionally reconstituted into the membrane of a single vesicle. Altogether, these data show the successful reconstitution of large DNA-origami pores into a lipid bilayer, paving the way to further developments from mimicking nuclear transport to recapitulating an artificial nucleus.\\[0.5pt]

\noindent To conclude, in \textbf{Chapter 8} we look back at the results provided in this thesis and propose new possible developments and follow-up projects that will further shed light on remaining unresolved issues. We first hypothesize that a simple framework that describes necessary and sufficient requirements for FG-Nup selectivity can be achieved by studying the behavior of designer FG-Nups upon systematic variations of the amino acid sequence. In light of the compelling evidence showing the participation of Karyopherins into establishing the selective barrier, we also point to new possible experiments and expansions of the current techniques to further our understanding of the complex Kap-Nup interaction. Next, we point to possible routes for exploiting the newly developed DNA-origami nanopore platform for studying transport through a potentially more controlled, tunable, and physiologically relevant FG-mesh. To conclude, we zoom out to the broad field of nuclear transport and provide our perspective on remaining outstanding questions. I am confident that tackling the current unresolved issues related to our understanding of FG-Nup phase separation, the interplay with Kaps, and the fuzzy architecture of NPC central channel will ultimately provide a satisfying mechanistic understanding of selective nuclear transport.

\chapter*{Samenvatting}
\addcontentsline{toc}{chapter}{Samenvatting}
\setheader{Samenvatting}

{\selectlanguage{dutch}

Op het eerste gezicht lijken nanopori\"{e}n misschien eenvoudig om te doorgronden, want het zijn (letterlijk) slechts heel kleine gaatjes in een membraan. En u zou zich zelfs kunnen afvragen waarom we \"{u}berhaupt zoveel hoogopgeleide wetenschappers nodig hebben om zulke eenvoudige objecten te bestuderen. Het korte antwoord is omdat nano\-poriën, zoals de naam al doet vermoeden, objecten zijn op de nanometer schaal, die men dus niet zo makkelijk kan zien of anderszins kan waarnemen. Het uitgebreide antwoord kunt u vinden in dit proefschrift, waarin ik mijn onderzoek aan vele verschillende typen van nano\-poriën beschrijf en bediscussieer. De typen variëren van biologische nano\-poriën tot vaste-stof nano\-poriën en DNA-origami nano\-poriën. Het centrale thema van mijn onderzoek is echter het moleculaire complex dat zich bevindt in de kernporie (nuclear pore complex, of kortweg `NPC' in het Engels): de nanoporie in het biologische membraan dat de celkern, met daarin het erfelijk DNA materiaal, scheidt van het cytoplasma.
Maar voordat ik daartoe kom zal ik u eerste meenemen naar de wondere wereld van ionen die door vaste-stof nano\-poriën stromen en de fluctuaties die zich daarin voordoen, beter bekend als stroomruis. Vervolgens komen we aan bij het concept van de biomimetische nano\-poriën, oftewel vaste-stof nano\-poriën die we in het laboratorium zo veel mogelijk laten lijken op biologische nano\-poriën, door ze te bedekken met bio\-logische membranen en/of biologische eiwitten. Hierdoor komt het gedrag van deze biomimetische nano\-poriën verbazingwekkend dicht in de buurt van dat van de echte kernporie in de cel. Deze biomimetische nano\-poriën stellen ons dus in staat om beter te doorgronden hoe de kernporiën van onze cellen zich gedragen, en bieden veel potentie voor nieuwe wetenschappelijke ontdekkingen.\\[0.5pt]
	
\noindent \textbf{Hoofdstuk 1} is de algemene inleiding op dit proefschrift, waarin de benodigde natuurkundige basisprincipes aan de orde komen, alsmede een kort overzicht van de opbouw van een biologische cel. Vervolgens zal ik inzoomen op de kernporie, door te beschrijven hoe deze werd ontdekt, hoe ze eruit ziet, en uit welke eiwitten ze is opgebouwd. Ook zal ik uitleggen hoe de zogenaamde FG-Nup eiwitten een cruciale poortwachter-achtige rol vervullen, waardoor alleen een selecte groep van biologische moleculen getransporteerd kan worden door de kernporie. Daarnaast zal ik in dit hoofdstuk de gangbare theorieën aanstippen voor het mechanisme van transport door de kernporie, waarbij de nadruk zal liggen op het `FG-centrische' model en het conflicterende `Kap-centrische' model.
Omdat wij in onze experimenten veel gebruik maken van nanotechnologie zal ik in dit hoofdstuk ook de methoden en technieken beschrijven waarmee we onze onderzoeksobjecten fabriceren, en hoe we één enkel molecuul kunnen detecteren. Tevens laat ik zien hoe vaste-stof nano\-poriën (net als de echte kernporie) bepaalde biologische moleculen kunnen selecteren voor transport, tenminste, wanneer we de porie bekleden met FG-Nup eiwitten. Dit hoofdstuk sluit ik vervolgens af met een beschrijving van de basisprincipes van DNA origami, en een overzicht van belangrijke wetenschappelijke artikelen die verschenen zijn op het gebied van DNA-origami nano\-poriën.\\[0.5pt]
	
	\noindent In \textbf{Hoofdstuk 2} vergelijk ik vaste-stof nano\-poriën met biologische nano\-poriën, dat wil zeggen, ik vergelijk hun algemene karakteristieken en hun stroomruis. Ik begin met het identificeren van de verschillende oorzaken van hoogfrequente en laagfrequente ruis. Want hoewel de oorzaak van de hoogfrequente ruis in beide systemen hetzelfde is, namelijk dielectrische ruis en capacitieve ruis, verschilt de oorzaak van de laagfrequente ruis: bij vaste-stof nano\-poriën is het vooral 1/f ruis, terwijl het bij biologische nano\-poriën vooral protonatieruis is. Vervolgens vergelijk ik de signaal/ruis verhouding (SRV) van verschillende nano\-poriën tijdens het transport van korte homopolymeren. Op basis van literatuurgegevens en data van zowel ons eigen lab als die van andere labs blijkt dat SiN\textsubscript{x} vaste-stof nano\-poriën de hoogste SRV hebben ($\sim$37). Dit komt doordat er in dit systeem relatief weinig laagfrequente ruis is, maar juist een relatief hoge stroomsterkte en dus een sterker signaal. Daarbij moet overigens opgemerkt worden dat de SRV van de biologische nanoporie MspA sterk verhoogd kan worden (>160$\times$) door een DNA-motor eiwit aan de ingang van deze nanoporie te plaatsen. Ik sluit dit hoofdstuk af met een overzicht van de beste methoden om zowel hoog- als laagfrequente stroomruis te onderdrukken in biologische- en vaste-stof nano\-poriën.\\[0.5pt]
	
	
	\noindent Tot op de dag van vandaag is er één bron van stroomruis die nanoporie wetenschappers niet goed begrijpen: de zogenaamde 1/f ruis. In \textbf{Hoofdstuk 3} karakteriseer ik het laagfrequente ruis-spectrum van vaste-stof nano\-poriën, en zien we dat de grootte van de fluctuaties in 1/f ruis afneemt bij grotere nano\-poriën. Hoewel deze interessante observatie niet nieuw is, bestond er nog geen theoretisch model om dit fenomeen op een analytische manier te verklaren. In dit hoofdstuk laat ik zien dat de globale trends van dit fenomeen goed te verklaren zijn door ook rekening te houden met de bijdrage van de nanoporie-ingang aan de 1/f ruis, hetgeen in eerdere modellen niet gedaan werd. Verder introduceren we twee verschillende `Hooge parameters' om twee lokale mechanismen voor ionenfluctuaties te beschrijven, enerzijds aan het oppervlakte van de nanoporie, en anderzijds in de bulk en de ingang van de nanoporie. Ons uiteindelijke model werkt hierdoor goed voor nano\-poriën van verschillende diameters, en vormt daarmee een veralgemenisering van het Hooge model voor vaste-stof nano\-poriën.\\[0.5pt]
	
	\noindent \textbf{Hoofdstuk 4} geeft een breed overzicht van het nano\-poriën onderzoeksveld, met de nadruk op toepassing die verder gaan dan het traditionele sequencen van DNA of RNA moleculen. Door belangrijke bijdragen van nano\-poriën in andere onderzoeksvelden te benadrukken laten we zien hoe veelzijdig en hoe breed toepasbaar zowel biologische als vaste-stof nano\-poriën kunnen zijn, en hoe ze daardoor kunnen bijdragen aan een breed scala van wetenschappelijke uitdagingen. De veelzijdigheid van nano\-poriën is te danken aan tenminste drie parameters die gevarieerd kunnen worden, zoals de lengte en de diameter van de porie, en de `bekleding' van het oppervlakte waardoor de interactie met andere moleculen aangepast kan worden. In dit overzichtshoofdstuk komen onder meer de volgende toepassingen aan de orde: proteomics met moleculaire resolutie, biopsie van biomarkers met moleculaire resolutie, het creëren van nanoreactoren voor polymeerchemie en het nabootsen van biologische poriën (zoals de kernporie) om een betreffend biologische transport proces te bestuderen.\\[0.5pt]
	
	\noindent In \textbf{Hoofdstuk 5} borduren we voort op eerder biomimetisch werk waarin de kernporie werd nagebootst door vaste-stof nano\-poriën te bekleden met gezuiverde FG-Nup eiwitten. Hier maken we echter geen gebruik meer van een uit de natuur afkomstige FG-Nup, maar van een synthetisch FG-Nup eiwit dat we zelf ontworpen hebben en `NupX' noemen. Onze hypothese is dat we de het gedrag van de kernporie zouden moeten kunnen nabootsen door kunstmatige `designer FG-Nup' eiwitten te ontwerpen die voldoen aan dezelfde bouwkundige principes als hun natuurlijke tegenhangers. Drie bouwkundige principes hebben we weten te achterhalen door de aminozuursequentie van de belangrijkste GLFG-Nups uit bakkersgist te analyseren: (i) een bi-modale verdeling van zowel geladen als plakkerige aminozuren, om precies te zijn: een plakkerig N-terminaal domein met weinig lading, en een niet-plakkerig C-terminaal domein met relatief veel lading; (ii) een herhaling van het `FG' of het `GLFG' motief met een onderliggende afstand van 5-20 aminozuren in het plakkerige gedeelte; en (iii) eiwitten die intrinsieke ongeordend zijn, dat wil zeggen dat ze geen vaste drie-dimensionale structuur hebben.
	Om de selectiviteit van ons designer eiwit NupX te onderzoeken hebben we een QCM-D chip (waarmee eiwit-eiwit interacties gemeten kunnen worden) ermee bekleed, en konden we aantonen dat de kernporie-transporter Kap95 op een vergelijkbare manier bindt aan NupX als aan een natuurlijke FG-Nup. Het controle eiwit BSA daarentegen (dat niet betrokken is bij kernporietransport) bindt noch aan NupX noch aan zijn natuurlijke tegenhanger. En toen we biomimetische nano\-poriën bekleedden met NupX zagen we dat Kap95 (in tegenstelling tot BSA) de stroom van ionen onderbreekt, hetgeen aantoont dat het inderdaad middels NupX getransporteerd kan worden. Door deze experimentele resultaten te combineren met computersimulaties van de conformatie en de domein lokalisatie van NupX hebben we een gedetailleerd microscopisch beeld kunnen krijgen van deze biomimetische nanoporie. Samengevat toont dit hoofdstuk dus aan dat kernporietransport in het laboratorium ook goed nagebootst kan worden met een rationeel ontworpen synthetisch eiwit.\\[0.5pt]
	
	
	\noindent In \textbf{Hoofdstuk 6} bestuderen we met biomimetische nano\-poriën de interactie tussen de kernporie-transporter Kap95 en een natuurlijk FG-Nup eiwit, namelijk Nsp1. In eerdere studies is weliswaar de selectiviteit van Nsp1 voor Kap95 aangetoond, maar in dit hoofdstuk wordt er pas voor het eerst gekeken naar de concentratieafhankelijkheid van dit transport, teneinde onderscheid te kunnen maken tussen twee conflicterende theoretische modellen: het Kap-centrische- en het FG- centrische model. Wanneer we een oplossing met Kap95 over een met Nsp1 beklede vaste-stof nanoporie spoelden namen we een gestage afname in de ionenstroom waar, hetgeen erop duidt dat een gedeelte van de Kap95 populatie aan Nsp1 bindt met een hoge affiniteit. Daarnaast observeerden we bovenop die gestage afname in de ionenstroom vele korte blokkades (van millisecondes) die duiden op een tweede populatie van Kap95 moleculen die snel getransporteerd worden. Een analyse van de laagfrequente ruis, die bij biomimetische nano\-poriën vooral gekenmerkt wordt door FG-Nup fluctuaties in de nanoporie, toonde aan dat de 1/f ruis afnam bij toenemende Kap95 concentraties. Dit is in overeenstemming met een verwachte versteviging van het FG-Nup netwerk in de nanoporie, die veroorzaakt wordt door de eerstgenoemde populatie van hoog-affiniteit gebonden Kap95 moleculen. We bevestigen onze nanoporie-metingen met QCM-D metingen, waar we ook een concentratie-afhankelijke toename van Kap95 in het Nsp1 netwerk waarnemen. Met andere woorden, uit verschillende metingen blijkt dat er zowel een langzame als een snelle Kap95 populatie is in het FG-Nup netwerk van de nanoporie, wat in overeenstemming is met voorspellingen van het zogeheten Kap-centrische model.\\[0.5pt]
	
	\noindent De zoektocht naar betere biomimetische kernporiën heeft wetenschappers naar DNA origami geleid, omdat dat ingezet kan worden als een drie-dimensionale ringvormige mal waaraan FG-Nup eiwitten op een gecontroleerde manier vastgemaakt kunnen worden. Tot nu toe was het echter niet mogelijk om de transporteigenschappen van FG-Nups binnenin zo’n origami mal te bestuderen. In \textbf{Hoofdstuk 7} beschrijven we een nieuw methode om relatief grote (30 nm) DNA origami-pori\"{e}n in te bouwen in het membraan van een liposoom, oftewel een vetblaasje. Teneinde de hoge energiebarrière (om een dusdanig groot object in een membraan te krijgen) te omzeilen hebben we gebruik gemaakt van de methode `cDICE'. Hiermee werden de DNA-origami pori\"{e}n al in het membraan ingebouwd op het moment dat dit gevormd wordt vanuit een omgekeerde emulsie. We hebben vervolgens geverifieerd dat de DNA-origamis inderdaad de verwachte poriën vormen door de influx van verschillende moleculen erdoorheen te meten. Hieruit bleek dat kleine (GFP van $\sim$30 kDa) tot middelgrote moleculen (dextran van $\sim$250 kDa) de DNA-origami porie gemakkelijk konden passeren, maar grotere dextran moleculen niet, precies zoals verwacht op basis van de porie diameter. Daarnaast hebben we het aantal DNA-origami poriën per liposoom gekwantificeerd door middel van FRAP experimenten, waaruit blijkt dat er honderden in \'{e}\'{e}n enkel liposoom ingebouwd kunnen worden. Samengevat tonen deze experimenten aan dat DNA-origami poriën zeer geschikt zijn als biomimetische kernporie om het transport van biologische moleculen over de kernmembraan te bestuderen.\\[0.5pt]
	
	\noindent In \textbf{Hoofdstuk 8} blik ik terug op de resultaten die in dit proefschrift beschreven zijn, en stel ik nieuwe projecten en ontwikkelingen voor die gevolgd zouden kunnen worden om onopgeloste problemen te benaderen. Zo stel ik voor om FG-Nup eiwitten in meer detail te bestuderen door de aminozuur samenstelling van synthetische designer FG-Nups systematisch te variëren, en het effect daarvan te bepalen op selectiviteit en transport. Ook stel ik nieuwe experimenten voor waarmee we de interacties tussen Kap-eiwitten en FG-Nups beter zouden kunnen karakteriseren, en hoe we op basis van DNA-origami biomimetische kernporiën zouden kunnen maken die dichter in de buurt komen van de natuurlijke kernporie. Als allerlaatste zoom ik uit op het gehele onderzoeksveld dat transport door de kernporie bestudeert, en geet ik mijn visie op openstaande thema’s zoals (i) de fase-scheiding van FG-Nup eiwitten; (ii) het samenspel tussen FG-Nups en Kap- eiwitten; en (iii) de drie-dimensionale structuur van de kernporie. Ik ben ervan overtuigd dat we een bevredigend mechanistisch inzicht in dit transport proces kunnen krijgen wanneer we deze verschillende thema’s gelijktijdig aanpakken.
	
}

